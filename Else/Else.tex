% Options for packages loaded elsewhere
\PassOptionsToPackage{unicode}{hyperref}
\PassOptionsToPackage{hyphens}{url}
\PassOptionsToPackage{dvipsnames,svgnames,x11names}{xcolor}
%
\documentclass[
  authoryear,
  preprint]{elsarticle}

\usepackage{amsmath,amssymb}
\usepackage{iftex}
\ifPDFTeX
  \usepackage[T1]{fontenc}
  \usepackage[utf8]{inputenc}
  \usepackage{textcomp} % provide euro and other symbols
\else % if luatex or xetex
  \usepackage{unicode-math}
  \defaultfontfeatures{Scale=MatchLowercase}
  \defaultfontfeatures[\rmfamily]{Ligatures=TeX,Scale=1}
\fi
\usepackage{lmodern}
\ifPDFTeX\else  
    % xetex/luatex font selection
\fi
% Use upquote if available, for straight quotes in verbatim environments
\IfFileExists{upquote.sty}{\usepackage{upquote}}{}
\IfFileExists{microtype.sty}{% use microtype if available
  \usepackage[]{microtype}
  \UseMicrotypeSet[protrusion]{basicmath} % disable protrusion for tt fonts
}{}
\makeatletter
\@ifundefined{KOMAClassName}{% if non-KOMA class
  \IfFileExists{parskip.sty}{%
    \usepackage{parskip}
  }{% else
    \setlength{\parindent}{0pt}
    \setlength{\parskip}{6pt plus 2pt minus 1pt}}
}{% if KOMA class
  \KOMAoptions{parskip=half}}
\makeatother
\usepackage{xcolor}
\setlength{\emergencystretch}{3em} % prevent overfull lines
\setcounter{secnumdepth}{5}
% Make \paragraph and \subparagraph free-standing
\makeatletter
\ifx\paragraph\undefined\else
  \let\oldparagraph\paragraph
  \renewcommand{\paragraph}{
    \@ifstar
      \xxxParagraphStar
      \xxxParagraphNoStar
  }
  \newcommand{\xxxParagraphStar}[1]{\oldparagraph*{#1}\mbox{}}
  \newcommand{\xxxParagraphNoStar}[1]{\oldparagraph{#1}\mbox{}}
\fi
\ifx\subparagraph\undefined\else
  \let\oldsubparagraph\subparagraph
  \renewcommand{\subparagraph}{
    \@ifstar
      \xxxSubParagraphStar
      \xxxSubParagraphNoStar
  }
  \newcommand{\xxxSubParagraphStar}[1]{\oldsubparagraph*{#1}\mbox{}}
  \newcommand{\xxxSubParagraphNoStar}[1]{\oldsubparagraph{#1}\mbox{}}
\fi
\makeatother


\providecommand{\tightlist}{%
  \setlength{\itemsep}{0pt}\setlength{\parskip}{0pt}}\usepackage{longtable,booktabs,array}
\usepackage{calc} % for calculating minipage widths
% Correct order of tables after \paragraph or \subparagraph
\usepackage{etoolbox}
\makeatletter
\patchcmd\longtable{\par}{\if@noskipsec\mbox{}\fi\par}{}{}
\makeatother
% Allow footnotes in longtable head/foot
\IfFileExists{footnotehyper.sty}{\usepackage{footnotehyper}}{\usepackage{footnote}}
\makesavenoteenv{longtable}
\usepackage{graphicx}
\makeatletter
\def\maxwidth{\ifdim\Gin@nat@width>\linewidth\linewidth\else\Gin@nat@width\fi}
\def\maxheight{\ifdim\Gin@nat@height>\textheight\textheight\else\Gin@nat@height\fi}
\makeatother
% Scale images if necessary, so that they will not overflow the page
% margins by default, and it is still possible to overwrite the defaults
% using explicit options in \includegraphics[width, height, ...]{}
\setkeys{Gin}{width=\maxwidth,height=\maxheight,keepaspectratio}
% Set default figure placement to htbp
\makeatletter
\def\fps@figure{htbp}
\makeatother

\makeatletter
\@ifpackageloaded{caption}{}{\usepackage{caption}}
\AtBeginDocument{%
\ifdefined\contentsname
  \renewcommand*\contentsname{Table of contents}
\else
  \newcommand\contentsname{Table of contents}
\fi
\ifdefined\listfigurename
  \renewcommand*\listfigurename{List of Figures}
\else
  \newcommand\listfigurename{List of Figures}
\fi
\ifdefined\listtablename
  \renewcommand*\listtablename{List of Tables}
\else
  \newcommand\listtablename{List of Tables}
\fi
\ifdefined\figurename
  \renewcommand*\figurename{Figure}
\else
  \newcommand\figurename{Figure}
\fi
\ifdefined\tablename
  \renewcommand*\tablename{Table}
\else
  \newcommand\tablename{Table}
\fi
}
\@ifpackageloaded{float}{}{\usepackage{float}}
\floatstyle{ruled}
\@ifundefined{c@chapter}{\newfloat{codelisting}{h}{lop}}{\newfloat{codelisting}{h}{lop}[chapter]}
\floatname{codelisting}{Listing}
\newcommand*\listoflistings{\listof{codelisting}{List of Listings}}
\makeatother
\makeatletter
\makeatother
\makeatletter
\@ifpackageloaded{caption}{}{\usepackage{caption}}
\@ifpackageloaded{subcaption}{}{\usepackage{subcaption}}
\makeatother
\journal{Introduction to Text Mining Using R}

\ifLuaTeX
  \usepackage{selnolig}  % disable illegal ligatures
\fi
\usepackage[]{natbib}
\bibliographystyle{elsarticle-harv}
\usepackage{bookmark}

\IfFileExists{xurl.sty}{\usepackage{xurl}}{} % add URL line breaks if available
\urlstyle{same} % disable monospaced font for URLs
\hypersetup{
  pdftitle={Are Rom-Coms Dead?},
  pdfauthor={Naren Prakash; Fathima Shaikh; Caleb Williams},
  colorlinks=true,
  linkcolor={blue},
  filecolor={Maroon},
  citecolor={Blue},
  urlcolor={Blue},
  pdfcreator={LaTeX via pandoc}}


\setlength{\parindent}{6pt}
\begin{document}

\begin{frontmatter}
\title{Are Rom-Coms Dead? \\\large{Project Proposal} }
\author[1]{Naren Prakash%
%
}

\author[1]{Fathima Shaikh%
%
}

\author[1]{Caleb Williams%
%
}


\affiliation[1]{organization={University of California, Los
Angeles, Department of Statistics and Data Science},city={Los
Angeles},postcode={90024},postcodesep={}}

\cortext[cor1]{Corresponding author}



        
\begin{abstract}
Romantic comedies, or ``rom-coms'', have been claimed to be worse now
than before. Is this truly the case? This project will use 30 popular
rom-com movie scripts from 1980 to 2020 in an attempt to answer the
question: Have romantic comedies declined in quality over time (past 40
years)? The methods employed are lexical diversity, topic modeling, and
sentiment analysis. The important features from these methods will help
to develop a random forest model that predicts the decade in which each
rom-com was published based on the original data and additional
engineered features.
\end{abstract}





\end{frontmatter}
    

\section{Introduction}\label{introduction}

Rom-coms have captivated audiences for decades, becoming a staple in
popular cinema. However, in recent years, critics and casual viewers
alike have argued that the quality of rom-coms has declined, referencing
factors such as weaker storytelling and formulaic plots. Additionally,
many claim that recent rom-coms are more predictable and cliche than
those from years before. This project explores whether or not rom-coms
have indeed declined in quality, using text mining techniques such as
lexical diversity, sentiment extraction, correlation analysis, and topic
modeling to examine transcripts of popular movies since 1980. Through
this approach, this paper aims to determine whether the perceived
decline in rom-com quality is supported by measurable changes in film
content, or if it is simply a result of changing public perceptions such
as the effect of nostalgia.

\section{Objective}\label{objective}

This paper seeks to answer whether romantic comedies have declined in
quality over time, specifically the last 40 years. Quality will be
defined using quantitative metrics.

\section{Research Questions}\label{research-questions}

\begin{itemize}
\tightlist
\item
  Are recent romantic comedies more correlated with each other than
  those of the past?
\item
  Have romantic comedy scripts become less complex (in terms of
  vocabulary)?
\item
  Can the era of a romantic comedy be easily predicted based on aspects
  of the script?
\end{itemize}

\section{Methods}\label{methods}

The methods used in this project aspire to quantify how rom-coms have
changed over time through different metrics. One test of this involves
lexical richness and linguistic complexity. This involves using
Type-Token Ratio (TTR), which measures the diversity of a script's
vocabulary, and readability scores, which evaluate the ease of
comprehension for each script.

Another type of method used will be sentiment analysis, specifically
using lexicons like AFINN, NRC, and Bing to analyze how each script's
tone changes over time. This will help to see if romantic comedies have
a more positive or more negative sentiment as well as note similarities
in sentiments across time periods.

Also, this project will employ Topic Modeling through Latent Dirichlet
Allocation (LDA) to find recurring topics in the scripts. LDA will
contribute to exploring whether or not rom-com themes have become more
formulaic in the past 40 years.

Additionally, correlation analysis will be employed to investigate
similarities in script content among romantic comedies of different time
periods.

Lastly, after important features have been engineered, a Random Forest
model will help illustrate the similarity or dissimilarity of romantic
comedies across time periods. In forming a classification model, the aim
is to evaluate if scripts from different time periods are truly
different in some aspect or if there is another factor not considered in
this paper.


\renewcommand\refname{References}
  \bibliography{bibliography.bib}



\end{document}
